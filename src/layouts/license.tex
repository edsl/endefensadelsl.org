\newpage
\thispagestyle{empty}

\begin{flushleft}

\begin{tabularx}{\textwidth}{c X}
  \raisebox{-\totalheight}{\includegraphics[height=5em]{images/logo_edsl.png}} &

  \hbox{\Large{En Defensa del Software Libre}}

  En Defensa del Software Libre es una revista de teoría sobre
  Software y Cultura Libres. Se edita en papel y se distribuye
  gratuita y libremente en formato digital. \\
\end{tabularx}

\vfill
\copyleft  \the\year -- En Defensa del Software Libre.
\url{https://endefensadelsl.org}

Salvo donde se exprese lo contrario, los artículos y la edición se
liberan bajo la Licencia de Producción de Pares.

\url{https://endefensadelsl.org/ppl_deed_es.html}

\end{flushleft}
\newpage

\chapter*{Licencia de Producción de
Pares}\label{licencia-de-producciuxf3n-de-pares}

\section*{Ud. es libre de}\label{ud.-es-libre-de}

\begin{itemize}
\item
  Compartir - copiar, distribuir, ejecutar y comunicar públicamente la
  obra
\item
  Hacer obras derivadas
\end{itemize}

\section*{Bajo las condiciones
siguientes:}\label{bajo-las-condiciones-siguientes}

\begin{center}
  \begin{tabularx}{\textwidth}{c X}
    \raisebox{-\totalheight}{\includegraphics[height=3em]{images/by.png}} &

    \textbf{Atribución} -- Debe reconocer los créditos de la obra de la
    manera especificada por el autor o el licenciante (pero no de una
    manera que sugiera que tiene su apoyo o que apoyan el uso que hace
    de su obra). \\

    \raisebox{-\totalheight}{\includegraphics[height=3em]{images/sa.png}} &

    \textbf{Compartir bajo la Misma Licencia} -- Si altera o transforma
    esta obra, o genera una obra derivada, sólo puede distribuir la obra
    generada bajo una licencia idéntica a ésta. \\

    \raisebox{-\totalheight}{\includegraphics[height=3em]{images/nc.png}} &

    \textbf{No Capitalista} -- La explotación comercial de esta obra
    sólo está permitida a cooperativas, organizaciones y colectivos sin
    fines de lucro, a organizaciones de trabajadores autogestionados, y
    donde no existan relaciones de explotación. Todo excedente o
    plusvalía obtenidos por el ejercicio de los derechos concedidos por
    esta Licencia sobre la Obra deben ser distribuidos por y entre los
    trabajadores. \\

  \end{tabularx}
\end{center}

\newpage

\section*{Entendiendo que}\label{entendiendo-que}

\begin{itemize}
\item
  \textbf{Renuncia} - Alguna de estas condiciones puede no aplicarse si
  se obtiene el permiso del titular de los derechos de autor.
\item
  \textbf{Dominio Público} - Cuando la obra o alguno de sus elementos se
  halle en el dominio público según la ley vigente aplicable, esta
  situación no quedará afectada por la licencia.
\item
  \textbf{Otros derechos} - Los derechos siguientes no quedan afectados
  por la licencia de ninguna manera:

  \begin{itemize}
  \item
    Los derechos derivados de usos legítimos u otras limitaciones
    reconocidas por ley no se ven afectados por lo anterior;
  \item
    Los derechos morales del autor;
  \item
    Derechos que pueden ostentar otras personas sobre la propia obra o
    su uso, como por ejemplo derechos de imagen o de privacidad.
  \end{itemize}
\item
  \textbf{Aviso} - Al reutilizar o distribuir la obra, tiene que dejar
  muy en claro los términos de la licencia de esta obra. La mejor forma
  de hacerlo es enlazar a esta página.
\end{itemize}

